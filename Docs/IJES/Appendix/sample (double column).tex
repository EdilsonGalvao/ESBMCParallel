%%%%%%%%%%%%%%%%%%%%%%
\documentclass{doublecol-new}
%%%%%%%%%%%%%%%%%%%%%%

\usepackage{natbib,stfloats}
\usepackage{mathrsfs,amsmath,upgreek}

\def\newblock{\hskip .11em plus .33em minus .07em}

\newcommand{\be}{\begin{eqnarray}}
\newcommand{\ee}{\end{eqnarray}}
\newcommand{\nn}{\nonumber}

\theoremstyle{TH}{
\newtheorem{lemma}{Lemma}
\newtheorem{theorem}[lemma]{Theorem}
\newtheorem{corrolary}[lemma]{Corrolary}
\newtheorem{conjecture}[lemma]{Conjecture}
\newtheorem{proposition}[lemma]{Proposition}
\newtheorem{claim}[lemma]{Claim}
\newtheorem{stheorem}[lemma]{Wrong Theorem}
\newtheorem{algorithm}{Algorithm}
}

\theoremstyle{THrm}{
\newtheorem{definition}{Definition}
\newtheorem{question}{Question}
\newtheorem{remark}{Remark}
\newtheorem{scheme}{Scheme}
}

\theoremstyle{THhit}{
\newtheorem{case}{Case}[section]
}

\makeatletter

\def\Reals{\mathbb{R}}
\def\Ints{\mathbb{Z}}
\def\Nats{\mathbb{N}}

\def\theequation{\arabic{equation}}

\def\tc{\textcolor{red}}

\def\BottomCatch{%
\vskip -10pt
\thispagestyle{empty}%
\begin{table}[b]%
\NINE\begin{tabular*}{\textwidth}{@{\extracolsep{\fill}}lcr@{}}%
\\[-12pt]
Copyright \copyright\ 2012 Inderscience Enterprises Ltd. & &%
\end{tabular*}%
\vskip -30pt%
%%\vskip -35pt%
\end{table}%
%\def\tc{\textcolor}{red}
} \makeatother

%%%%%%%%%%%%%%%%%
\begin{document}%
%%%%%%%%%%%%%%%%%

\thispagestyle{plain}

\setcounter{page}{1}

\LRH{xxxx}

\RRH{xxxx}

\VOL{x}

\ISSUE{x}

\PUBYEAR{xxxx}

%\BottomCatch

%\CLline

%\subtitle{}

\title{xxxx}

\authorA{xxxx}

\affA{xxxx*\\
*Corresponding author}

\authorB{xxxx}
\affB{xxxx}

\begin{abstract}
xxxxx
\end{abstract}

\KEYWORD{xxxx}

\REF{to this paper should be made as follows: xxxx (xxxx) `xxxx',
{\it xxxx}, Vol.~x, No.~x, pp.xxx--xxx.}

\begin{bio}
\tc{AUTHOR PLEASE SUPPLY CAREER HISTORY OF NO MORE THAN 100 WORDS
FOR EACH AUTHOR.}
\end{bio}

\maketitle

\section{Introduction}
\label{sec-introduction}

\section{Performance evaluation}
\label{sec-performance}


\section{Related work}
\label{sec-related}


\section{Conclusions}
\label{sec-conclusion}


\section*{Acknowledgements}



\begin{thebibliography}{99}

\bibitem[\protect\citeauthoryear{xxxx}{2011}]{Zhang11}
xxxx (2011) `Reliable packets delivery over segment-based multi-path
in wireless ad hoc networks', {\it Pervasive Computing and
Applications (ICPCA), 2011 6th International Conference on},
pp.300-�306. \tc{AUTHOR PLEASE SUPPLY LOCATION.}

\end{thebibliography}


\section*{Appendx}

We have proposed the CSD scheme that makes use of multiple bridge
nodes within the close neighbourhood of the source and insecure
neighbours. These bridge nodes cooperatively help to deliver secret
link key information from the source towards the insecure
neighbours. In the delivery process, no secret information is
disclosed to any node en route. These routers simply forward the
encrypted data. Even the bridge nodes only get to know partial
information about the secrets that they help to deliver.

\end{document}

%\def\notesname{Note}
%
%\theendnotes

%\section*{Query}
%
%\tc{AQ1: AUTHOR PLEASE CITE FIGURE 8 IN TEXT.}

\end{document}
