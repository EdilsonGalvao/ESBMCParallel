%%%%%%%%%%%%%%%%%%%%%%% file template.tex %%%%%%%%%%%%%%%%%%%%%%%%%
%
% This is a general template file for the LaTeX package SVJour3
% for Springer journals.          Springer Heidelberg 2010/09/16
%
% Copy it to a new file with a new name and use it as the basis
% for your article. Delete % signs as needed.
%
% This template includes a few options for different layouts and
% content for various journals. Please consult a previous issue of
% your journal as needed.
%
%%%%%%%%%%%%%%%%%%%%%%%%%%%%%%%%%%%%%%%%%%%%%%%%%%%%%%%%%%%%%%%%%%%
%
% First comes an example EPS file -- just ignore it and
% proceed on the \documentclass line
% your LaTeX will extract the file if required
\begin{filecontents*}{example.eps}
%!PS-Adobe-3.0 EPSF-3.0
%%BoundingBox: 19 19 221 221
%%CreationDate: Mon Sep 29 1997
%%Creator: programmed by hand (JK)
%%EndComments
gsave
newpath
  20 20 moveto
  20 220 lineto
  220 220 lineto
  220 20 lineto
closepath
2 setlinewidth
gsave
  .4 setgray fill
grestore
stroke
grestore
\end{filecontents*}
%
\RequirePackage{fix-cm}
%
%\documentclass{svjour3}                     % onecolumn (standard format)
%\documentclass[smallcondensed]{svjour3}     % onecolumn (ditto)
\documentclass[twocolumn]{svjour3}       % onecolumn (second format)
%\documentclass[twocolumn]{svjour3}          % twocolumn
%
\smartqed  % flush right qed marks, e.g. at end of proof
%
\usepackage{graphicx}
%
% \usepackage{mathptmx}      % use Times fonts if available on your TeX system
%
% insert here the call for the packages your document requires
%\usepackage{latexsym}
% etc.
%
% please place your own definitions here and don't use \def but
% \newcommand{}{}
%
% Insert the name of "your journal" with
% \journalname{myjournal}
%

\begin{document}
\title{Applying Multi-Core Model Checking to Hardware-Software Partitioning in Embedded Systems}

\author{Alessandro Trindade\and Hussama Ismail\and Renato Degelo\and Edilson Galvao\and Lucas Cordeiro}

\institute{F. Author \at
              first address \\
              Tel.: +123-45-678910\\
              Fax: +123-45-678910\\
              \email{fauthor@example.com}           %  \\
%             \emph{Present address:} of F. Author  %  if needed
           \and
           S. Author \at
              second address
}

\date{Received: date / Accepted: date}
\maketitle

\begin{abstract}
We present an alternative approach to solve the hardware (HW) and software (SW) partitioning problem, which uses Bounded Model Checking (BMC) based on Satisfiability Modulo Theories (SMT) in conjunction with a multi-core support using Open MultiProcessing. The multi-core SMT-based BMC approach allows initializing many verification instances based on processors cores numbers available to the model checker. Each instance checks for a different optimum value until the optimization problem is satisfied. The goal is to show that multi-core model-checking techniques can be effective, in particular cases, to find the optimal solution of the HW-SW partitioning problem using an SMT-based BMC approach. We compare the experimental results of our proposed approach with Integer Linear Programming and the Genetic Algorithm.
\keywords{hardware-software co-design \and hardware-software partitioning\and optimization\and model checking\and multi-core\and OpenMP }
\end{abstract}

\section{Introduction}
Nowadays, with the strong development of embedded systems, the design phase plays an important role. At early stages, the design is split into separated flows: hardware and software. Consequently, the partitioning decision process, which deals with the decisions upon which parts of the application have to be designed in hardware (HW) and which in software (SW), must be supported by any well-structured methodology. If not, this leads to a number of issues (design flow interruptions, redesigns, and undesired iterations) which affects the overall development process, the quality and the lifecycle of the final system. Starting at the 1990s, intensive research was performed, and several approaches proposed, as shown in [1] and [2].
In any HW and SW design of complex systems, more time is spent on verification than on construction [3]. Formal methods based on model checking offer great potential to obtain a more effective and faster verification in the design process. Programs may be viewed as mathematical objects with behavior that is, in principle, well determined. This makes it possible to specify programs using mathematical logic, which constitutes the intended (correct) behavior. Then, one can try to give a formal proof or otherwise establish that the program meets its specification [4]. Research in formal methods has led to the development of very promising verification techniques, which facilitate the early detection of errors. Model-based verification techniques use models that describe the possible system behavior in a mathematically precise and unambiguous manner. The system models are accompanied by algorithms that systematically explore all the states of the system model.
In [5] and [6] was shown that it is possible to use Bounded Model Checking (BMC) based on Satisfiability Modulo Theories (SMT) to perform HW-SW partitioning in embedded systems. The present work extends those studies since there is a substantial improvement in terms of the genetic algorithm and the SMT-based verification method, which has been extended with a multi-core architecture. Multi-core processors have been used in all segments of industry to implement high-performance computing [7]. In particular, hardware platforms, together with multi-processing platforms, have allowed verification algorithms to distribute tasks executions across multiple processors, which generate an increase in performance if compared to single-core solution. However, most verification algorithms still disregard the limitations of the CMOS technology, which limits the increase of the chip’s frequency after it reaches 4 GHz.
Here, we exploit the availability of multi-core processors. In particular, a multi-core SMT-based BMC method is applied to the HW-SW partitioning and then is compared to the results with classical integer linear programming (ILP) and genetic algorithm (GA) using a multi-core tool as well. To the best of our knowledge, this is the first work to use a multi-core SMT-based verification to solve a HW-SW partitioning problem in embedded systems. We implement our ideas with the Efficient SMT-based Bounded Model Checker (ESBMC) tool [14]. As its main contribution, this paper shows that it is possible to take advantage of an SMT-based BMC tool in a multi-core architecture to solve optimization problems.
This paper is organized as follows: Section II gives a background on optimization, model checking, and multi-core support with Open MultiProcessing. Section III describes informal and formal mathematical modeling. Section IV describes briefly the binary integer programming and GA algorithms. The SMT-based BMC method is presented in Section V. Section VI presents the experimental evaluation. Section VII discusses related work. Section VIII presents the conclusion and future work.


\section{BACKGROUND}
\subsection{Optimization}
Optimization is the act of obtaining the best result (i.e., the optimal solution) under given circumstances [9]. In the design, construction, and maintenance of any engineering system, engineers have to make many technological and managerial decisions at several stages. The ultimate goal of all such decisions is either to minimize the effort required or to maximize the desired benefit. Because the effort required or the benefit desired in any practical situation can be expressed as a function of certain decision variables, optimization can be defined as the process of finding the conditions that give the maximum or minimum value of a function [9].
There is no single method available for solving all optimization problems efficiently [9]. The most known technique is linear programming, which is an method applicable for the solution of problems in which the objective function and the constraints appear as linear functions of the decision variables. A particular case of linear programming is ILP, in which the variables can assume just integer values. Eq. (1) shows a typical linear programming problem, where  and  are vectors or matrixes that describe the constraints.

\subsection{Optimization with Vz}
\subsection{Bounded Model Checking with ESBMC}
\subsection{Multi-core ESBMC with OpenMP}
\subsection{Multi-core ESBMC with OpenMP using Binary Search}

\section{Mathematical modeling}
The mathematical modeling was taken from [1], [2].
\subsection{Informal Model (or Assumptions)}
\subsection{Formal Model}

\section{Partitioning problem using ILP-based,Genetic Algorithms}
\label{ILPGA}
The ILP and GA were taken from [5] and [6]. Both use slack variables in order to be possible to represent the constraints and to use commercial tools. However, GA had improvements from the parameters of related studies in order to increase the solution accuracy without producing timeout. The tuning was performed by empirical tests and resulted in changing of three parameters, which are passed to function ga of MATLAB [18]: the population size was set from 300 to 500, the Elite count changed from 2 (default value) to 50, and the number of Generations changed from 100* NumberOfVariables (default) to 75.

\section{Analysis of the partitioning problem using ESBMC}
\section{Analysis of the partitioning problem using vZ}
\section{Experimental Evaluation}
\section{Related Work}
\section{Conclusions}
\section{References}

\end{document}


% end of file template.tex

